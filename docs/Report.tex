% Created 2022-10-21 Fri 00:31
% Intended LaTeX compiler: pdflatex
\documentclass[a4paper]{article}
\usepackage[utf8]{inputenc}
\usepackage[T1]{fontenc}
\usepackage{graphicx}
\usepackage{longtable}
\usepackage{wrapfig}
\usepackage{rotating}
\usepackage[normalem]{ulem}
\usepackage{amsmath}
\usepackage{amssymb}
\usepackage{tikz}
\usepackage{capt-of}
\usepackage{hyperref}
\author{Vishal Kalathil}
\date{\today}
\title{White Box Neural Network}
\hypersetup{
 pdfauthor={Vishal Kalathil},
 pdftitle={White Box Neural Network},
 pdfkeywords={},
 pdfsubject={ANN},
 pdfcreator={Emacs 28.2 (Org mode 9.6)}, 
 pdflang={English}}
\usepackage{biblatex}

\begin{document}

\maketitle



\section{Introduction}
In this report I will go over my experiences with implementing a whitebox ANN in Rust.
For this task I decided to create a simple neural network that would attempt to recognize the digits from the MNIST dataset.

\section{Usage}
Run
\begin{verbatim}
  cargo run Cargo.toml
\end{verbatim}
after installing rust and cargo

\section{Implementation}
The ANN reads data from a modified version of the MNIST dataset that was converted into a file for easier reading. I also trained the model on the test dataset (which consisted of 10000 entries and tested the performance using the first 200 elements in the train data set, as I was running short of computational resources)

The ANN also has:
\begin{itemize}
\item 28\textsuperscript{2} nodes in the input layer
\item 10 nodes in the hidden layer
\item 10 nodes in the output layer
\end{itemize}
I also ended up using 2 different activation function combinations to see which one performed better.
The first was the sigmoid function and the other was a combination of the ReLU function and the Softmax function.
\newpage
\section{Performance}
At 1000 epochs with learning rate = 0.2
\begin{longtable}{|l|l|l|}
\caption{Accuracy metrics} \label{tab:long} \\

\hline \multicolumn{1}{|c|}{\textbf{Activation Function}} & \multicolumn{1}{c|}{\textbf{Accuracy on Train Data}} & \multicolumn{1}{c|}{\textbf{Accuracy on Test Data}} \\ \hline
\endfirsthead

\multicolumn{3}{c}%
{{\bfseries \tablename\ \thetable{}}} \\
\hline \multicolumn{1}{|c|}{\textbf{}} & \multicolumn{1}{c|}{\textbf{Accuracy on Training Data`}} & \multicolumn{1}{c|}{\textbf{Accuracy on Test Data}} \\ \hline
\endhead


\hline
\endlastfoot

Sigmoid & 0.9213 & 0.915 \\
ReLU + Softmax & abcdef ghjijklmn & 123.456778 \\
\end{longtable}
\section{Final Thoughts}
Despite the many, many places this project can be improved upon like
\begin{enumerate}
\item Multithreading
\item Cleaner and more optimized code
\item Fix all the warnings
\item Build to a crate (?)
\item Use the actual dataset
\item Use GPU throught the RustCUDA project
\end{enumerate}

The neural networks do perform decently well.
\end{document}
